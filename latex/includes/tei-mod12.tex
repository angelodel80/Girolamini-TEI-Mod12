\begin{frame}
    \frametitle{TEI Modulo 12 - Codifica Edizioni Critiche}
    \framesubtitle{Getting started}
    \addtocounter{nframe}{1}
    
    \begin{block}{Main Discipline}
    \end{block}

    \begin{block}{Main Goal}
    \end{block}

\end{frame}

\begin{frame}
    \frametitle{Introduction}
    \addtocounter{nframe}{1}
    
    \begin{center}
        \includegraphics[width=.95\textwidth]{imgs/1.png}
    \end{center}

\end{frame}

% This panel addresses the TEI critical apparatus as a data model, investigating how it has expanded the capacity of scholarly editions to articulate and analyze phenomena of textual variation and multiplicity. We will discuss how the TEI critical apparatus, as a structure that mediates distinct versions of a work, is expanding horizons for multidimensional and pluralistic document modeling. Our panel surveys recent experiments with the critical apparatus that have led to new kinds of scholarly research and in some cases to revisions to the TEI Guidelines. What kinds of research questions and applications can we support with the TEI critical apparatus, and what practical challenges do we face in working with it in inline and stand-off ways? We begin by investigating how the TEI critical apparatus has transformed the expressive capacity of scholarly editions to prioritize textual multiplicity. We continue by sharing data models that apply TEI critical apparatus as a stand-off “spine” for connecting independently encoded witnesses. We conclude by inviting the audience to discuss with us the scalability of these methods for texts with large numbers of witnesses, and the technological challenges and opportunities of stand-off methods in light of recent changes to the TEI Guidelines. 

