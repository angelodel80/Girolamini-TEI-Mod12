% Seminario Girolamini 17 giugno 2020
%  temi:
% - recensio/raccolta e descrizione dei testimoni
% - collatio/esaminatio
% - costitutio/emendatio
% - pubblicazione/messa in pagina
%
% - Intro XML <https://www.tei-c.org/release/doc/tei-p5-doc/en/html/SG.html>
%
% - Intro TEI < https://books.openedition.org/oep/426 >
%
% - TEI Modulo 12 < https://www.tei-c.org/release/doc/tei-p5-doc/en/html/TC.html >
%
% - Tool di allineamento per la collazione:
%
% -- http://v-machine.org/documentation/
%
% -- https://collatex.net/
%
% -- http://www.juxtacommons.org/
%
% -- https://sada.uzi.uni-halle.de/
%
% -- medite
%
% -- tustep/TXSTEP
%
% -- http://evt.labcd.unipi.it/

% -- http://www.informatik.uni-leipzig.de:8080/BibleViz/

% ecdotica: resa grafica, messa in pagina
% -- https://ride.i-d-e.de/issues/issue-11/reledmac/
% -- https://tei-c.org/2014/06/10/tei-critical-edition-toolbox/
% -- http://teicat.huma-num.fr/index.php

\documentclass{beamer}
    
%    \usepackage[english]{babel}
    %\usepackage[latin1]{inputenc}
    %\usepackage[T1]{fontenc}

\mode<presentation>{
  \setbeamertemplate{background canvas}[vertical shading]
  \usetheme{Berkeley}
  \useoutertheme{himinfolines}
}
  
\usepackage{ucs}
\usepackage[utf8]{inputenc}
\usepackage[english,polutonikogreek,italian,UKenglish,british]{babel}
\usepackage{graphicx}
\usepackage{colortbl}
\usepackage{multicol}
\usepackage{ulem}
\usepackage{verbatim}
\usepackage{alltt}
\usepackage{ccicons}
\usepackage{MnSymbol,wasysym}
\usepackage{tikzsymbols}
\usepackage{textcomp}
\usepackage{xmpincl}

\usepackage{parskip}
\setcounter{nframes}{100}
\setcounter{nframe}{1}
\setbeamercovered{dynamic}
\newenvironment{grcenv}{\begin{otherlanguage}{greek}}{\end{otherlanguage}}
\newcommand{\g}[1]{\textgreek{#1}}
\definecolor{darkgreen}{rgb}{0,0.5,0}
\definecolor{darkblue}{rgb}{0,0,0.5}
\definecolor{grey}{rgb}{0.5,0.5,0.5}
\setcounter{tocdepth}{5}

\makeatletter

\makeatother
%\includexmp{LicencesAndLicensing}

%frame00 metadata
	\title{Dalla Recensio all'Emendatio Digitale}
	\author[A.M. Del Grosso]{Angelo Mario Del Grosso}
	%\institute{}
	\institute{\texttt{angelo.delgrosso@ilc.cnr.it} \\\bigskip\textit{CNR-ILC} \\\bigskip\url{http://ilc.cnr.it/}}
    \institute{\texttt{angelo.delgrosso@ilc.cnr.it} \\\bigskip\textbf{Teoria, Prassi e Strumenti}}
    \date{Istituto di Linguistica Computazionale ``A. Zampolli'', \today}
    \AtBeginSection[]{
    \begin{frame}<beamer>
    \addtocounter{nframe}{1}
    \footnotesize
    \frametitle{Progress status}
    \tableofcontents[currentsection,hideothersubsections]
    \end{frame}
    }

\begin{document}

\begin{frame}
	\maketitle
\end{frame}

\begin{frame}
	\frametitle{Argomenti trattati}
	\tableofcontents
\end{frame}

\section{Presentazione}

\begin{frame}
	\frametitle{Di cosa mi occupo}
	\addtocounter{nframe}{1}

	\begin{block}{Filologia Digitale e Computazionale}
		Attività di ricerca per lo sviluppo di sistemi di linguistica e filologia digitale e computazionale volti alla produzione, rappresentazione, analisi, fruizione e interrogazione di testi di tradizione medievale, a stampa e di autori moderni e contemporanei.
	\end{block}

\end{frame}

\begin{frame}
	\frametitle{Profilo professionale e di ricerca}
	\addtocounter{nframe}{1}

	\begin{block}{In sintesi}
		\begin{center}
			Ingegnere Informatico prestato alla filologia computazionale
		\end{center}
	\end{block}

	\begin{center}
		\includegraphics[width=.7\textwidth]{imgs/InfrastructureForTextualScholarship.png}
	\end{center}

\end{frame}

\begin{frame}
	\frametitle{Temi del seminario}
	\addtocounter{nframe}{1}

	\begin{itemize}
		\item Filologia del testo 
		\begin{itemize}
			\item disciplina storica (insieme di discipline) dedicata al recupero/ricostruzione di un'opera - per lo più letteraria - e allo studio critico delle sue testimonianze condotto con metodo scientifico.
		\end{itemize}
		\item Metodo di Lachman
		\item Apparato Critico
		\begin{itemize}
			\item Apparato critico: nella edizione critica di un testo, l'apparato critico è il luogo (che può essere: a piè di pagina, in appendice al testo oppure dopo la nota al testo) in cui l'editore accoglie - a volte discutendolo (e comunque offrendo la possibilità di verifica del suo lavoro critico) - il complesso delle annotazioni, correzioni e varianti portate dalla tradizione e da lui giudicate erronee (o non meritevoli di essere accolte come lezione a testo).	
		\end{itemize}
		\item Collazione: collation is "to compare one text with another to discover textual variation"
		\item 
		\item 
	\end{itemize}

\end{frame}

% Apparato critico negativo: l'apparato negativo o implicito non riporta la lezione a testo ma solo la lezione o le lezioni non accolte dell'altro o degli altri testimoni rappresentati con sigle distintive. Sta al lettore identificare nel testo la lezione corrispondente.
%Apparato critico positivo: l'apparato positivo o esplicito riporta la lezione a testo (in genere solo la prima e l'ultima parola di brani lunghi) non di rado delimitata da una parentesi quadra ( ] ), seguita dalla lezione o lezioni dell'altro o degli altri testimoni rappresentati con sigle distintive.

\begin{frame}
	\frametitle{Temi del seminario}
	\addtocounter{nframe}{1}

	\begin{itemize}
		\item Codifica del testo 
		\begin{itemize}
			\item Rappresentazione formale con tecnologia digitale
			\item Livello di rappresentazione dei caratteri
			\item Livello di rappresentazione della struttura e dei fenomeni testuali
		\end{itemize}
	\end{itemize}

\end{frame}

\begin{frame}
	\frametitle{Temi del seminario}
	\addtocounter{nframe}{1}

	\begin{center}
		\includegraphics[width=.5\textwidth]{../imgs/tei-r.pdf}
	\end{center}

\end{frame}

\begin{frame}
	\frametitle{Obiettivo del seminario}
	\addtocounter{nframe}{1}

	\begin{block}{approfondiremo}
		Tecniche per la rappresentazione digitale di edizioni digitali adottando le specifiche e le linee guida della Text Encoding Initiative implementate in eXtensible Markup Language (TEI-XML).
	\end{block}
	\begin{block}{approfondiremo}
		Strumenti per la collazione automatica e la pubblicazione di edizioni critiche digitali
	\end{block}

\end{frame}

\begin{frame}
    \frametitle{Perché è importante la codifica dei testi}
    \framesubtitle{Motivazioni pratiche}
    \addtocounter{nframe}{1}
    
    \begin{block}{Perché codificare i testi}
        Per rendere disponibile l'immenso patrimonio testuale tramite l'uso di sistemi digitali e computazionali è necessario effettuare una trasposizione/transcodifica\textsuperscript{*} dei testi dal loro supporto originario verso il nuovo supporto elettronico (\textit{Machine Readable Form} and \textit{Machine Actionable Form}).
    \end{block}

    \begin{center}
        * \textit{procedimento di conversione dei dati codificati secondo un sistema verso un sistema diverso}
    \end{center}

\end{frame}

\begin{frame}
	\frametitle{Elementi di Codifica dei Caratteri}
	\framesubtitle{Definizioni}
	\addtocounter{nframe}{1}

	\begin{block}{Rappresentare il testo in formato digitale}
		L’adozione di metodologie e tecnologie informatiche per il trattamento di documenti testuali richiede in primo luogo la disponibilità di un'adeguato sistema di rappresentazione digitale dei dati - presenti nella risorsa originale - e conseguentemente un formalismo adatto a tale rappresentazione.
	\end{block}

\end{frame}

\begin{frame}
	\frametitle{Elementi di Codifica del testo}
	\framesubtitle{Formalismi}
	\addtocounter{nframe}{1}

	\begin{block}{Formati e formalismi di codifica}

		Ogni pezzo di informazione aggiunta ad un testo grezzo attraverso l'inserimento di dati metatestuali (markup, annotazione, codifica), constituisce il risultato di una analisi e di una interpretazione che è stata condotta (da un umano o da una macchina) al fine di esplicitare e rappresentare nel modo più accurato e completo possibile le informazioni da veicolare attraverso il formato digitale prescelto (anche in modo incrementale).


	\end{block}

\end{frame}

\section{Panoramica Text Encoding Initiative}
%\begin{frame}
	\frametitle{Markup language e XML}
	\framesubtitle{soluzione corrente per la codifica dei testi}
	\addtocounter{nframe}{1}

	\begin{block}{TEI-XML}
		Considerato ad oggi lo standard de facto per la codifica dei testi è lo schema XML messo a punto dalla Text Encoding Initiative (TEI-XML).
	\end{block}

\end{frame}

\begin{frame}
	\frametitle{TEI-XML}
	\framesubtitle{Motivazioni per adottare TEI}
	\addtocounter{nframe}{1}

	\begin{block}{Perché TEI}
		La Text Encoding Initiative (TEI) è un autorevole progetto internazionale, a cui afferiscono varie organizzazioni e università, il cui scopo è fornire agli studiosi di informatica umanistica uno strumento il più espressivo e flessibile possibile per rappresentare qualsiasi aspetto di interesse relativo alla risorsa testuale da rappresentare digitalmente.
	\end{block}

\end{frame}

\begin{frame}
	\frametitle{Introduzione}
	\addtocounter{nframe}{1}
    
    \begin{block}{Qual è l'obiettivo della TEI}
        L'obiettivo della TEI è quello di fornire linee guida per la creazione e la gestione in forma digitale di qualsiasi tipo di dato creato e usato in ambito umanistico.
        \\ E per questo motivo il consorzio investe molte risorse per la accessibilità e la divulgazione della tecnologia che da anni sviluppa.
    \end{block}
    
\end{frame}

\begin{frame}
	\frametitle{I principi fondamentali della TEI}
	\addtocounter{nframe}{1}
    
    \begin{center}
	    \includegraphics[width=.2\textwidth]{imgs/tei-r.pdf}
	\end{center}

    \begin{itemize}
        
        \item<1-> Le linee guida della TEI privilegiano il ``significato'' (meaning) del testo piuttosto che l'``aspetto'' (layout); privileggia il modello del testo, piuttosto che il formato.
          
        \item<2-> La TEI è stata progettata per essere indipendente dagli strumenti software che la usano per la creazione oppure per l'elaborazione dei documenti elettronici.

        \item<3-> La TEI cresce, matura, si evolve sulla base delle indicazioni e delle ricerche dalla propria comunità di riferimento (community-driven).
           
    \end{itemize}
    
\end{frame}

%%Codifica di testi

% Le norme TEI
% Roberto Rosselli Del Turco
% Dipartimento di Studi Umanistici
% Università di Torino
% roberto.rossellidelturco@fileli.unipi.it
% roberto.rossellidelturco@unito.itLa codifica di testi – Le norme TEI
% La Text Encoding Initiative
% Sito WWW: http://www.tei-c.org/

\begin{frame}
	\frametitle{Intro Text Encoding Initiative}
	\framesubtitle{TEI}
	\addtocounter{nframe}{1}

	\begin{block}{Motto}
		TEI: Yesterday's information tomorrow
	\end{block}

	\begin{block}{Dal sito TEI}
		“an international and interdisciplinary standard that
		enables libraries, museums, publishers, and individual
		scholars to represent a variety of literary and linguistic
		texts for online research, teaching, and preservation”
	\end{block}
\end{frame}


\begin{frame}
	\frametitle{Intro Text Encoding Initiative}
	\framesubtitle{TEI}
	\addtocounter{nframe}{1}

	\begin{block}{Testo di riferimento}
        Guidelines for Electronic Text Encoding and Interchange 
        \\( \url{http://www.tei-c.org/Guidelines/} )
	\end{block}

	\begin{block}{testo di ausilio}
		BURNARD, Lou. What is the Text Encoding Initiative? How to add intelligent markup to digital resources. Nouva edizione [online]. Marseille: OpenEdition Press, 2014 (creato il 13 octobre 2018). Disponibile su Internet: \url{http://books.openedition.org/oep/426}. ISBN: 9782821834606. DOI: 10.4000/books.oep.426.

	\end{block}
\end{frame}


\begin{frame}
	\frametitle{Intro Text Encoding Initiative}
	\framesubtitle{TEI}
	\addtocounter{nframe}{1}

	\begin{block}{un po' di storia}
		\begin{itemize}
			\item 1987: necessità di standard che permetta la creazione e l’interscambio di documenti per mezzo di archivi informatici(convegno NY)
			\item 1990: prima versione delle Guidelines (TEI P1)
			\item 1990-94: fondi garantiti da enti quali NEH, Mellon Foundation, la Comunità Europea; supporto di ACH, ACL, ALLC
		\end{itemize}
	\end{block}

\end{frame}

\begin{frame}
	\frametitle{Intro Text Encoding Initiative}
	\framesubtitle{TEI}
	\addtocounter{nframe}{1}

	\begin{block}{un po' di storia}
		\begin{itemize}
			\item 2000: nascita del TEI Consortium, associazione non profit per lo sviluppo dello standard TEI
			\item 2002: passaggio da SGML a XML con la v. P4
			\item 2007: nuova versione TEI P5, continuamente aggiornata
		\end{itemize}
	\end{block}

\end{frame}


\begin{frame}
    \frametitle{Intro Text Encoding Initiative}
    \framesubtitle{TEI}
    \addtocounter{nframe}{1}

	\begin{block}{TEI Guidelines}
		\textit{versioni P1 e P3 basate su SGML}
		\\\textbf{versione P4}
		\begin{itemize}
			\item standard precedente, ancora impiegata
			\item basata su XML, DTD tradizionale
			\item pubblicata in forma definitiva nel 2002
			\item \url{http://www.tei-c.org/Guidelines/P4/}
		\end{itemize}
    \end{block}
    
\end{frame}

\begin{frame}
	\frametitle{Intro Text Encoding Initiative}
	\framesubtitle{TEI}
	\addtocounter{nframe}{1}

	\begin{block}{TEI Guidelines: versione P5}
		\begin{itemize}
			\item basata su XML, schema RelaxNG (e DTD tradizionale)
			\item pubblicata alla fine del 2007, aggiornata due volte l’anno
			\item molte novità interessanti (in particolare: maggior modularità)
			\item  http://www.tei-c.org/Guidelines/P5/
		\end{itemize}
	\end{block}

\end{frame}

\begin{frame}
	\frametitle{Intro Text Encoding Initiative}
	\framesubtitle{TEI}
	\addtocounter{nframe}{1}

	\begin{block}{TEI Guidelines: Obiettivi}
		\begin{itemize}
			\item better interchange and integration of scholarly data
			\item support for all texts, in all languages, from all periods
			\item guidance for the perplexed: what to encode - hence, a user-driven codification of existing best practice
			\item assistance for the specialist: how to encode --- hence, a loose framework into which unpredictable extensions can be fitted
		\end{itemize}
	\end{block}

\end{frame}



\begin{frame}
	\frametitle{Intro Text Encoding Initiative}
	\framesubtitle{TEI}
	\addtocounter{nframe}{1}

	\begin{block}{TEI Guidelines: Obiettivi}
		These apparently incompatible goals result in a highly flexible,
		modular, environment for DTD customization.
		Lou Burnard, TEI and XML: a marriage made in heaven?
		(http://www.tei-c.org/Talks/marriage.xml?style=printable)
	\end{block}

\end{frame}


\begin{frame}
	\frametitle{Intro Text Encoding Initiative}
	\framesubtitle{TEI}
	\addtocounter{nframe}{1}

	\begin{block}{Che cosa offre la TEI}
        \begin{itemize}
            \item un ricco (e complesso) manuale di codifica, 
            \\(le Guidelines for Electronic Text Encoding and Interchange)
            \item un numero elevato di elementi (sia strutturale sia semantico)
            \item schemi di codifica
            \item infrastruttua modulare e personalizzabile
        \end{itemize}

	\end{block}

\end{frame}

\begin{frame}
	\frametitle{Intro Text Encoding Initiative}
	\framesubtitle{TEI}
	\addtocounter{nframe}{1}

	\begin{block}{Che cosa offre la TEI}
		\begin{itemize}
			\item \textbf{è possibile scegliere soltanto i moduli necessari}
			\item \textbf{è possibile modificare le definizioni degli elementi}
		\end{itemize}

	\end{block}

\end{frame}

\begin{frame}
	\frametitle{Intro Text Encoding Initiative}
	\framesubtitle{TEI}
	\addtocounter{nframe}{1}

	\begin{block}{Supporto per gli utenti}
		\begin{itemize}
			\item il sito del consorzio (\url{http://www.tei-c.org/})
			\item pagine relative alle varie versioni delle Guidelines
			\item software, tutorial, ecc.
			\item il wiki: \url{http://www.tei-c.org/wiki/index.php/Main_Page}
			\item la mailing list TEI\-L
			\item Github: \url{https://github.com/TEIC}
			\item TEI by example: \url{http://teibyexample.org/}
		\end{itemize}
	\end{block}
\end{frame}


\begin{frame}
	\frametitle{Intro Text Encoding Initiative}
	\framesubtitle{TEI}
	\addtocounter{nframe}{1}

	\begin{block}{Novità della versione P5}
		\begin{itemize}
			\item modulo di descrizione dei manoscritti
			\item grafica e multimedia
			\item standoff markup
			\item supporto per i namespace XML
			\item miglioramenti nel modulo feature structure
		\end{itemize}

	\end{block}

\end{frame}

% \begin{frame}
% 	\frametitle{Intro Text Encoding Initiative}
% 	\framesubtitle{TEI}
% 	\addtocounter{nframe}{1}

% 	\begin{block}{Novità della versione P5 (cont.)}
% 		\begin{itemize}
% 			\item modulo di descrizione dei manoscritti
% 			\item grafica e multimedia
% 			\item standoff markup
% 			\item supporto per i namespace XML
% 			\item miglioramenti nel modulo feature structure
% 		\end{itemize}

% 	\end{block}

% \end{frame}

\begin{frame}
	\frametitle{Intro Text Encoding Initiative}
	\framesubtitle{TEI}
	\addtocounter{nframe}{1}

	\begin{block}{TEI: Struttura modulare}
		\begin{itemize}
			\item si scelgono soltanto i moduli che corrispondono alle proprie esigenze, in modo da realizzare rapidamente uno schema di codifica appropriato
			\item ogni modulo contiene un certo numero di elementi (tagset)
			\item gli elementi sono organizzati in classi (strutturali, semantiche)
			\item gli attributi sono organizzati in classi (globali e specifici)
		\end{itemize}

	\end{block}

\end{frame}


% classi strutturali: elementi che hanno un ruolo a livello strutturale
% (paragrafi, strofe, etc.)
% classi semantiche: elementi descrittivi del testo (aggiunte,
% correzioni, nomi di persona, etc.)
% anche gli attributi sono organizzati in classi
% attributi globali: disponibili per tutti gli elementi
% attributi specifici: solo per alcuni elementi


\begin{frame}
	\frametitle{Intro Text Encoding Initiative}
	\framesubtitle{TEI}
	\addtocounter{nframe}{1}

	\begin{block}{TEI: Struttura modulare - Moduli essenziali}
		\begin{itemize}
			\item \textbf{tei}: definisce le classi di elementi, le macro e i datatype che verranno usati per tutti i moduli
			\item \textbf{header}: l’intestazione contenente i metadati relativi al documento TEI XML
			\item \textbf{textstructure}: elementi strutturali per qualsiasi tipo di testo
			\item \textbf{core}: elementi utili in qualsiasi tipo di documento
		\end{itemize}

	\end{block}

\end{frame}

\begin{frame}
	\frametitle{Intro Text Encoding Initiative}
	\framesubtitle{TEI}
	\addtocounter{nframe}{1}

	\begin{block}{TEI: Struttura modulare - Moduli facoltativi}
		\begin{itemize}
			\item \textbf{analysis}: strumenti per analisi (linguistica etc.) del testo
			\item \textbf{corpus}: gestione di corpora linguistici
			\item \textbf{drama}: elementi per testi teatrali e drammatici
			\item \textbf{gaiji}: rappresentazione di caratteri e glifi non standard
			\item \textbf{msdescription}: metadati relativi a manoscritti
		\end{itemize}

	\end{block}

\end{frame}


\begin{frame}
	\frametitle{Intro Text Encoding Initiative}
	\framesubtitle{TEI}
	\addtocounter{nframe}{1}

	\begin{block}{TEI: Struttura modulare - Moduli facoltativi}
		\begin{itemize}
			\item \textbf{spoken}: trascrizione del parlato
			\item \textbf{textcrit}: apparato critico
			\item \textbf{transcr}:  trascrizione di fonti primarie (manoscritti)
			\item \textbf{verse}: elementi supplementari per testi poetici
		\end{itemize}

	\end{block}

\end{frame}

\begin{frame}
	\frametitle{Intro Text Encoding Initiative}
	\framesubtitle{TEI}
	\addtocounter{nframe}{1}

	\begin{block}{TEI Lite}

		una specifica \textbf{personalizzazione} della TEI versione P4/5

	\end{block}

	\textit{A simple demonstration of how the TEI encoding scheme
		might be adopted to meet 90\% of the needs of 90\% of the
		TEI user community (dalla Prefatory Note: \\url{http://www.tei-
		c.org/Lite/})}

\end{frame}


\begin{frame}
	\frametitle{Intro Text Encoding Initiative}
	\framesubtitle{TEI}
	\addtocounter{nframe}{1}

	\begin{block}{TEI Lite}

		La \textbf{versione P4} è stata tradotta anche in italiano: TEI Lite:
		introduzione alla codifica dei testi, a cura di \textit{F. Ciotti} (
		\url{http://www.tei-c.org/Lite/teiu5_it.html}).
    \end{block}
    \textit{Molto usata, ma presenta varie limitazioni, con la versione P5 è più semplice produrre una versione semplificata o personalizzata}
\end{frame}

\begin{frame}
	\frametitle{Intro Text Encoding Initiative}
	\framesubtitle{TEI}
	\addtocounter{nframe}{1}

	\begin{block}{TEI Pizza chef}

		la TEI P4 poteva essere modificata usando un programma su
		web chiamato Pizza chef
		\url{http://www.tei-c.org.uk/pizza.html}.
		\\Metafora della base e dei condimenti (toppings),
		corrispondenti ai moduli indispensabili e a quelli facoltativi.
		\\Meccanismo efficace, soprattutto considerando l’alternativa
		(modifica manuale delle DTD TEI).
    \end{block}
    
    \textit{Tool obsoleto, necessaria comunque modifica manuale per nuovi elementi}


\end{frame}

\begin{frame}
	\frametitle{Intro Text Encoding Initiative}
	\framesubtitle{TEI}
	\addtocounter{nframe}{1}

	\begin{block}{TEI Roma}

		Per la P5 si è deciso di proporre una modularizzazione più
		efficace, e uno strumento di personalizzazione più potente.
		Strumento basato sul nuovo formato ODD.


	\end{block}

	\begin{block}{TEI Roma}
		Il metodo da seguire per la versione P5 (quella che
		useremo) fino all’arrivo del successore (Byzantium)

	\end{block}

\end{frame}

\begin{frame}
	\frametitle{Intro Text Encoding Initiative}
	\framesubtitle{TEI}
	\addtocounter{nframe}{1}

	\begin{block}{TEI Roma}

		\begin{itemize}
			\item  possibilità di scegliere, escludere, modificare sia gli elementi (e le classi di elementi), sia gli attributi (e le classi di attributi)
			\item possibilità di aggiungere elementi (eventualmente inserendoli nelle classi preesistenti)
			\item possibilità di salvare lo schema in tre formati diversi: DTD  tradizionale, W3C e RelaxNG (anche in forma compatta)
		\end{itemize}

	\end{block}


\end{frame}


\begin{frame}
	\frametitle{Intro Text Encoding Initiative}
	\framesubtitle{TEI}
	\addtocounter{nframe}{1}

	\begin{block}{TEI Roma: Il formato ODD}
		le versioni P1 – P4 delle DTD TEI erano nel formato DTD language e dipendevano dalla sintassi SGML per molti aspetti
	\end{block}

	\begin{block}{TEI Roma: Il formato ODD}
		One Document Does it all (ODD): set di specifiche in base alle quali un semplice documento TEI XML \textit{“defines a schema in terms of the modules it requires, together with any possible modifications, such as the desired
        root element”}. 
    \end{block}
    
    \textbf{Risultato finale: lo schema nel linguaggio desiderato e la relativa documentazione}

\end{frame}


\begin{frame}
	\frametitle{Intro Text Encoding Initiative}
	\framesubtitle{TEI}
	\addtocounter{nframe}{1}

	\begin{block}{TEI Roma: Il formato ODD}

		tutorial: Getting Started with P5 ODDs\\
		\url{http://www.tei-c.org/Guidelines/Customization/odds.xml}

	\end{block}


\end{frame}


% 15La codifica di testi – Le norme TEI
% TEI Roma 2
% per motivi vari, la manutenzione di TEI Roma lascia a
% desiderare (per non parlare dello sviluppo di un successore)
% nel passato recente la componente di “controllo” dello
% schema (Sanity checker) non funzionava
% in tal caso è possibile andare sul sito TEI di Oxford dove è in
% esecuzione una versione più vecchia:
% http://tei.it.ox.ac.uk/Roma/
% Roma, in ogni caso, è solo un front-end per il linguaggio ODD
% (cfr. infra)




\begin{frame}
	\frametitle{Intro Text Encoding Initiative}
	\framesubtitle{TEI}
	\addtocounter{nframe}{1}

	\begin{block}{Il futuro della TEI}
		\textbf{negli ultimi anni gli schemi TEI sono stati oggetto di alcune critiche}
	\end{block}

	\begin{block}{Il futuro della TEI}
		\begin{itemize}
			\item la TEI è troppo grande / complicata / piccola
			\item la TEI è basata su XML e questo formato è in declino
			\item la TEI/XML non supporta le gerarchie multiple
			\item la TEI non supporta il markup di tipo stand-off
		\end{itemize}
	\end{block}

\end{frame}


\begin{frame}
	\frametitle{Intro Text Encoding Initiative}
	\framesubtitle{TEI}
	\addtocounter{nframe}{1}

	\begin{block}{Il futuro della TEI}
		la cosa importante da ricordare è che il formato XML non è
		la TEI (in passato SGML, in futuro chissà → abstraction level)
		al contrario, se c’è una discrepanza fra Guidelines e schemi,
		la precedenza va alle Guidelines
	\end{block}

\end{frame}


\begin{frame}
	\frametitle{Intro Text Encoding Initiative}
	\framesubtitle{TEI}
	\addtocounter{nframe}{1}

	\begin{block}{text encoding con la TEI}
		E' caldamente raccomandato usare direttamente la
		versione più recente della P5.\\
		La flessibilità della P5 permette di definire uno schema di
		codifica che corrisponda precisamente al modello
    \end{block}
    
    \textit{La comunità di utenti e sviluppatori TEI offre un buon supporto.}

\end{frame}




%%%%%%%%%%%%%%%%%%%%%%
% Codifica di testi
% I moduli base della TEI
% Roberto Rosselli Del Turco
% Dipartimento di Studi Umanistici
% Università di Torino
% roberto.rossellidelturco@fileli.unipi.it
% roberto.rossellidelturco@unito.itSchemi di codifica TEI – Moduli base
% Elementi disponibili per tutti i documenti TEI

\begin{frame}
	\frametitle{Intro Text Encoding Initiative}
	\framesubtitle{TEI}
	\addtocounter{nframe}{1}

	\begin{block}{Un documento TEI P5 ‘minimo’}
        \begin{itemize}
            \item prologo XML
            \item intestazione TEI
            \item elementi strutturali
            \item elementi semantici dei moduli base
        \end{itemize}
    \end{block}
    
\end{frame}

\begin{frame}
	\frametitle{Intro Text Encoding Initiative}
	\framesubtitle{TEI}
	\addtocounter{nframe}{1}

   \textbf{ Moduli di base: \textit{tei}, \textit{header}, \textit{textstructure}, \textit{core}}

	\begin{block}{un documento TEI P5 ‘minimo’}
        Anche usando soltanto i moduli essenziali si ha a disposizione
        uno schema adatto alla marcatura di numerosi tipi di testi.
    \end{block}
 
       \textit{Schemi ``leggeri'' consigliati: la TEI Lite, o se necessario una
        versione più ridotta della P5 (TEI
        Absolutely Bare)}
\end{frame}



\begin{frame}
	\frametitle{Intro Text Encoding Initiative}
	\framesubtitle{Schemi di codifica TEI – Moduli base}
	\addtocounter{nframe}{1}

	\begin{block}{Caratteristiche degli elementi illustrati}
        \begin{itemize}
            \item gli elementi TEI rientrano nelle categorie generali di
            elementi XML che abbiamo visto
            \item elementi che possono contenere solo altri elementi (=
            elementi strutturali)
            \item elementi che possono contenere altri elementi e testo
            \item elementi che possono contenere solo testo
            \item  elementi vuoti (es. \texttt{<pb/>})
            \item  gli elementi vuoti marcano una gerarchia differente
        \end{itemize}
    \end{block}
\end{frame}


\begin{frame}
	\frametitle{Intro Text Encoding Initiative}
	\framesubtitle{Gerarchie multiple}
	\addtocounter{nframe}{1}

	\begin{center}
		\includegraphics[width=.95\textwidth]{imgs/overlap.png}
	\end{center}	
	
        % \texttt{<?xml version="1.0" encoding="utf-8"?>
        % <text>
        % <titolo>Gli assassinii della Rue Morgue</titolo>
        % <intestazione> I </intestazione>
        % \emph{<pagina n=``5''>}
        % <p>Le facoltà mentali che si sogliono chiamare analitiche sono, di
        % per se stesse, poco suscettibili di analisi [...]</p>
        % \emph{<p>}La facoltà di risolvere è probabilmente molto rinfor-
        % \emph{</pagina>}
        % <pagina n=``6''>
        % zata dallo studio delle matematiche e in modo particolare
        % dell’altissimo ramo di questa scienza che[...] \emph{</p>}
        % </pagina>
		% </text>}

\end{frame}



\begin{frame}
	\frametitle{Intro Text Encoding Initiative}
	\framesubtitle{Schemi di codifica TEI – Gerarchie multiple}
	\addtocounter{nframe}{1}

	\begin{center}
		\includegraphics[width=.95\textwidth]{imgs/overlap-b.png}

        % \texttt{<?xml version="1.0" encoding="utf-8"?>
        % <text>
        % <titolo>Gli assassinii della Rue Morgue</titolo>
        % <intestazione> I </intestazione>
        % <pagina n="5"/>
        % <p>Le facoltà mentali che si sogliono chiamare analitiche sono, di
        % per se stesse [...]</p>
        % <p>La facoltà di risolvere è probabilmente molto rinfor-
        % <pagina n="6"/>
        % zata dallo studio delle matematiche e in modo particolare
        % dell’altissimo ramo di questa scienza che [...] </p> </text>}
       
    \end{center}
    
   

\end{frame}

\begin{frame}
	\frametitle{Intro Text Encoding Initiative}
	\framesubtitle{Schemi di codifica TEI – Moduli base}
	\addtocounter{nframe}{1}

	\begin{block}{Struttura di un documento TEI}
        \begin{itemize}
            \item \textit{struttura fondamentale all’interno della radice (\texttt{<TEI>})}
            \item una intestazione TEI (\texttt{<teiHeader>})
            \item un testo: \texttt{<text>} (o più testi, cfr. infra)
        \end{itemize}
    \end{block}
    
\end{frame}


\begin{frame}
	\frametitle{Intro Text Encoding Initiative}
	\framesubtitle{Schemi di codifica TEI – Moduli base}
	\addtocounter{nframe}{1}

    \begin{block}{Contenuto del TEI header}
        \begin{itemize}
            \item metadati relativi al documento (utili per collezioni di testi
            codificati)
            \item descrizione del file usando \texttt{<fileDesc>} (obbligatoria)
            \item descrizioni relative al tipo di codifica, al contenuto del
            documento, alle sue revisioni (facoltative)
        \end{itemize}
    \end{block}
\textit{E' possibile includere testi introduttivi e spiegazioni relative alla
codifica effettuata (preziosi per l’interscambio!)}

\end{frame}


\begin{frame}
	\frametitle{Intro Text Encoding Initiative}
	\framesubtitle{Schemi di codifica TEI:  Moduli base}
	\addtocounter{nframe}{1}

	\begin{center}

		\includegraphics[width=.95\textwidth]{imgs/esempio1.png}
    %    \texttt{<?xml version="1.0" encoding="utf\-8"?>
    %      <!DOCTYPE TEI SYSTEM ``tei\_lite.dtd''>
    %      <TEI xmlns=``http://www.tei\-c.org/ns/1.0''>
    %      <teiHeader> </teiHeader>
    %      <text>
    %          <div>
    %          <p></p>
    %          </div>
    %     </text>
	% 	</TEI>}
    \end{center}
\end{frame}


\begin{frame}
	\frametitle{Intro Text Encoding Initiative}
	\framesubtitle{Schemi di codifica TEI – Moduli base}
	\addtocounter{nframe}{1}

	\begin{center}
		\includegraphics[width=.95\textwidth]{imgs/esempio2.png}
        % \texttt{<?xml version="1.0" encoding="utf-8"?>
        % <?xml-model href="tei-lite.rng"?>
        % <TEI xmlns=``http://www.tei-c.org/ns/1.0''>
        % <teiHeader>...</teiHeader>
        % <text>
        % <div>
        % <p></p>
        % </div>
        % </text>
        % </TEI>}
    \end{center}
    
   

\end{frame}

\begin{frame}
	\frametitle{Intro Text Encoding Initiative}
	\framesubtitle{Schemi di codifica TEI – Moduli base}
	\addtocounter{nframe}{1}

	\begin{block}{documento TEI - schema di intestazione TEI minima}
        Metadati essenziali riguardo il titolo, la modalità di diffusione e
        la fonte originaria di un testo codificato.
        \\Permettono classificazione, archiviazione ed elaborazione
        bibliografica
    \end{block}
    
\end{frame}

\begin{frame}
	\frametitle{Intro Text Encoding Initiative}
	\framesubtitle{Schemi di codifica TEI – Intestazione TEI minima}
	\addtocounter{nframe}{1}

	\begin{center}
		\includegraphics[width=.8\textwidth]{imgs/tei-header.png}
        % \texttt{<teiHeader>
        % <fileDesc>
        % <titleStmt>...</titleStmt>
        % <publicationStmt>...</publicationStmt>
        % <sourceDesc>...</sourceDesc>
        % </fileDesc>
        % </teiHeader>}
    \end{center}

\end{frame}




\begin{frame}
	\frametitle{Intro Text Encoding Initiative}
	\framesubtitle{Schemi di codifica TEI – Moduli base}
	\addtocounter{nframe}{1}

	\begin{center}
		\includegraphics[width=.8\textwidth]{imgs/header2.png}
	\end{center}
	
        % \texttt{<teiHeader>
        % <fileDesc>
        % <titleStmt>
        % <title>La Divina Commedia: versione elettronica</title>
        % <respStmt>
        % <resp>Conversione TEI P5 a cura di</resp><name>M. Rossi</name>
        % </respStmt>
        % </titleStmt>
        % <publicationStmt>
        % <publisher>Università di Pisa</publisher>
        % <date>2002-11-07</date>
        % <availability status=``restricted''><p></p></availability>
        % </publicationStmt>
        % <sourceDesc>
        % <bibl><title>La Divina Commedia</title><author>Dante Alighieri
        % </author><publisher>Mondadori</publisher>
        % <date>1988</date></bibl>
        % </sourceDesc>
        % </fileDesc>
        % </teiHeader>}

\end{frame}



\begin{frame}
	\frametitle{Intro Text Encoding Initiative}
	\framesubtitle{Schemi di codifica TEI – Moduli base}
	\addtocounter{nframe}{1}

    \begin{block}{Le altre componenti dell’intestazione TEI}
        \begin{itemize}
            \item \texttt{<encodingDesc>} informazioni riguardo lo schema (e il
            modello di codifica) utilizzato
            \item  \texttt{<profileDesc>} descrizione del testo: quando è stato
            creato, da chi, usando quali lingue etc.
            \item \texttt{<revisionDesc>} informazioni sulle versioni del file
        \end{itemize}
    \end{block}
    \textit{I metadati sono una componente essenziale di qualsiasi
        progetto di digitalizzazione}
\end{frame}


\begin{frame}
	\frametitle{Intro Text Encoding Initiative}
	\framesubtitle{Schemi di codifica TEI – Moduli base}
	\addtocounter{nframe}{1}

	\begin{block}{Elementi strutturali}
        
       \begin{itemize}
           \item \texttt{<text>} un singolo testo di qualsiasi tipo (punto di partenza della gerarchia).
           \item \texttt{<facsimile>} riproduzione della fonte primaria, può affiancare o sostituire \texttt{<text>}
           \item \texttt{<front>} figlio di \texttt{<text>} materiale che precede il testo
           \item \texttt{<body>} figlio di \texttt{<text>} rappresenta il testo stesso
           \item \texttt{<back>} figlio di \texttt{<text>} materiale che segue il testo
           \item \texttt{<group>} figlio di \texttt{<text>} alternativo a \texttt{<body>}, raggruppa testi diversi
       \end{itemize}
        
    \end{block}
    
   

\end{frame}



\begin{frame}
	\frametitle{Intro Text Encoding Initiative}
	\framesubtitle{Schemi di codifica TEI – Moduli base}
	\addtocounter{nframe}{1}


	\begin{center}
		\includegraphics[width=.95\textwidth]{imgs/esempio3.png}
	\end{center}

	% \begin{block}{Esempio schematico di documento TEI}
    %     \texttt{<TEI>
    %     <teiHeader> [informazioni del TEI Header]
    %     </teiHeader>
    %     <text>
    %     <front> [premessa, dedica ...] </front>
    %     <body> [corpo del testo ...] </body>
    %     <back> [postfazione, appendice ...]</back>
    %     </text>
    %     </TEI>}
    % \end{block}
    
   

\end{frame}



\begin{frame}
	\frametitle{Intro Text Encoding Initiative}
	\framesubtitle{Schemi di codifica TEI – Moduli base}
	\addtocounter{nframe}{1}
    \begin{block}{Costruire documenti compositi}
        \begin{itemize}
            \item rimpiazzando il \texttt{<body>} con un gruppo (\texttt{<group>}) di testi si ottiene un documento composito
            \item ciascuno di questi testi è rappresentato secondo una struttura
            standard
            \item un’altra possibilità è creare un corpus con \texttt{<teiCorpus>}
            \item intestazioni (\texttt{<teiHeader>}) separate per il corpus e per
            ciascun gruppo di testi
            \item struttura più complessa, su più livelli
        \end{itemize}
    \end{block}
\end{frame}


\begin{frame}
	\frametitle{Intro Text Encoding Initiative}
	\framesubtitle{Schemi di codifica TEI – Moduli base}
	\addtocounter{nframe}{1}

	\begin{center}
		\includegraphics[width=.95\textwidth]{imgs/composito1.png}
	\end{center}

        % \texttt{<TEI>
        % <teiHeader> [ intestazione del testo composito ] </teiHeader>
        % <text>
        % <front> [ front matter del composito ] </front>
        % <group>
        % <text>
        % <front> [ front matter del primo testo ] </front>
        % <body> [ body del primo testo ]
        % </body>
        % <back> [ back matter del primo testo ] </back>
        % </text>
        % <text>
        % <front> [ front matter del secondo testo] </front>
        % <body> [ body del secondo testo ]
        % </body>
        % <back> [ back matter del secondo testo ] </back>
        % </text>
        % ...
        % [ altri testi o gruppi di testi ]
        % ...
        % </group>
        % <back>
        % [ back matter del composito ]
        % </back>
        % </text>
        % </TEI>}

\end{frame}

\begin{frame}
	\frametitle{Intro Text Encoding Initiative}
	\framesubtitle{Schemi di codifica TEI – Moduli base}
	\addtocounter{nframe}{1}
	\begin{center}
		\includegraphics[width=.95\textwidth]{imgs/composito2.png}
	\end{center}

%     \begin{block}{Costruzione di corpora TEI}
%         \texttt{<teiCorpus>
% <teiHeader> [metadati per il corpus] </teiHeader>
% <TEI>
% <teiHeader> [metadati relativi al I testo]</teiHeader>
% <text> [primo testo del corpus] </text>
% </TEI>
% <TEI>
% <teiHeader>[metadati relativi al II testo]</teiHeader>
% <text> [secondo testo del corpus] </text>
% </TEI>
% </teiCorpus>}
%     \end{block}
\end{frame}


% \begin{frame}
% 	\frametitle{Intro Text Encoding Initiative}
% 	\framesubtitle{TEI}
% 	\addtocounter{nframe}{1}

% 	\begin{block}{TEI}
% 	 Schemi di codifica TEI – Moduli base
%   IMMAGINE
%     \end{block}
% \end{frame}


\begin{frame}
	\frametitle{Intro Text Encoding Initiative}
	\framesubtitle{Schemi di codifica TEI – Moduli base}
	\addtocounter{nframe}{1}

	\begin{block}{Altri elementi strutturali fondamentali}
        \begin{itemize}
            \item suddivisioni del testo, non numerati: \texttt{<div> }(nessun limite di nidificazione)
            \item suddivisioni del testo, numerati: \texttt{<div1> ... <div7> }(massimo 7 livelli)
            \item paragrafi: \texttt{<p>}
            \item testo riferito: \texttt{<q>} (discorso diretto, citazioni, etc.)
        \end{itemize}
        
    \end{block}
\end{frame}


\begin{frame}
	\frametitle{Intro Text Encoding Initiative}
	\framesubtitle{Schemi di codifica TEI – Moduli base}
	\addtocounter{nframe}{1}

	\begin{block}{Altri elementi strutturali fondamentali}
        \begin{itemize}
            \item versi: strofe \texttt{<lg>} e singoli versi \texttt{<l>}
            \item testi teatrali: discorsi \texttt{<sp>} che possono contenere paragrafi
            \texttt{<p>} o versi \texttt{<l>}, oltre a direzioni di scena \texttt{<stage>}
            \item milestone tags: \texttt{<pb/>, <lb/>, <cb/>, <milestone/>}
            \item notare che un \texttt{<div>} può contenere un \texttt{<floatingText>} (possibilità di introdurre gerarchie complesse).
        \end{itemize}
        
    \end{block}

\end{frame}

\begin{frame}
	\frametitle{Intro Text Encoding Initiative}
	\framesubtitle{Schemi di codifica TEI – Moduli base}
	\addtocounter{nframe}{1}

	\begin{block}{Apertura e chiusura di un \texttt{<div>}}
        \begin{itemize}
            \item \texttt{<head>}: qualunque tipo di intestazione: il titolo di un opera, l’intestazione di un paragrafo, di una sezione, ecc.
            \item l'attributo \textit{type} permette di classificare in base a una tipologia
            \item \texttt{<epigraph>} citazione all’inizio del testo, o nella pagina del titolo, eventualmente con riferimento bibliografico
            \item \texttt{<opener>} raggruppa un serie di elementi (data, luogo, saluti, ecc.) all’inizio del \texttt{<div>}, specie di una lettera
        \end{itemize}
    \end{block}

\end{frame}

\begin{frame}
	\frametitle{Intro Text Encoding Initiative}
	\framesubtitle{Schemi di codifica TEI – Moduli base}
	\addtocounter{nframe}{1}

	\begin{block}{Apertura e chiusura di un \texttt{<div>}}
        \begin{itemize}
            \item \texttt{<argument>}: lista degli argomenti trattati nel \texttt{<div>}
            \item \texttt{<trailer>} frase che compare alla fine del \texttt{<div>} (ad esempio ``Fine del capitolo 1'')
            \item \texttt{<closer>} raggruppa un serie di elementi (data, luogo, saluti, etc.) alla fine del \texttt{<div>}, specie di una lettera
        \end{itemize}
    \end{block}

\end{frame}


\begin{frame}
	\frametitle{Intro Text Encoding Initiative}
	\framesubtitle{Schemi di codifica TEI – Moduli base}
	\addtocounter{nframe}{1}        
	   
	\begin{center}
		\includegraphics[width=.9\textwidth]{imgs/esLettera.png}
	\end{center}

	% \texttt{<div type="lettera”>
    %     <opener>
    %     <dateline>
    %     <name type=``place''>Pisa</name>
    %     <date>20 marzo 2015</date>
    %     </dateline>
    %     <salute>Gentilissima Prof.ssa Scannagatti,</salute>
    %     </opener>
    %     <p>sono spiacente di doverle comunicare che un’invasione di cavallette si
    %     è abbattuta sui miei quaderni incautamente lasciati in giardino, e li ha
    %     divorati interamente.</p>
    %     <p>Questo purtroppo significa che non posso mostrare i compiti svolti,
    %     come sempre, con solerzia e assiduo impegno.</p>
    %     <closer>
    %     <salute>Certo di poter contare sulla sua comprensione le porgo i miei
    %     migliori saluti,</salute>
    %     <signed>Pierino Rossi</signed>
    %     </closer>
    %     </div>}

\end{frame}





\begin{frame}
	\frametitle{Intro Text Encoding Initiative}
	\framesubtitle{Schemi di codifica TEI – Moduli base}
	\addtocounter{nframe}{1}

	\begin{block}{Errori frequenti}
        \textit{Si fraintende il significato dell’elemento \texttt{<fileDesc>}}
        \begin{itemize}
            \item serve in primo luogo a dare informazioni sul file stesso, non sul testo originale
            \item il riferimento alla fonte dalla quale è tratto il testo codificato
            deve essere inserito nel \texttt{<sourceDesc>}
        \end{itemize}
    \end{block}

\end{frame}


\begin{frame}
	\frametitle{Intro Text Encoding Initiative}
	\framesubtitle{Schemi di codifica TEI – Moduli base}
	\addtocounter{nframe}{1}

	\begin{block}{Errori frequenti}
        I titoli sono codificati con \texttt{<title>} soltanto nel caso di riferimenti bibliografici i titoli del testo, dei capitoli etc. si marcano con \texttt{<head>}
    \end{block}
    
   

\end{frame}


\begin{frame}
	\frametitle{Intro Text Encoding Initiative}
	\framesubtitle{Schemi di codifica TEI – Moduli base}
	\addtocounter{nframe}{1}

	\begin{block}{Nota sugli errori possibili}
        \textbf{Tre categorie:}
        \begin{itemize}
            \item \textbf{errori sintattici}: un elemento inserito in un punto sbagliato
            della gerarchia, o che non può contenere testo etc.
            \item \textbf{errori di marcatura semantica}: usare un elemento inadatto
            allo scopo, ad esempio marcare un titolo con <emph>
            \item \textbf{errori di interpretazione} del testo (che portano al II tipo o
            all’assenza del markup che andrebbe inserito)
        \end{itemize}
       
    \end{block}
    \textit{Gli errori del primo tipo sono i più facili da individuare e
        correggere, quelli del terzo i più difficili}

\end{frame}


% \begin{frame}
% 	\frametitle{Intro Text Encoding Initiative}
% 	\framesubtitle{Schemi di codifica TEI – Moduli base}
% 	\addtocounter{nframe}{1}

% 	\begin{block}{ A proposito di <div>}
%         domanda che ricorre periodicamente: perché non usare nomi
%         più significativi invece di un contenitore generico come <div>
%         (= ‘suddivisione, parte di un testo’)?
%         perché non usare <book>, <chapter>, <section>, etc. come
%         fa, ad esempio, lo schema DOCBOOK?
%         troppa variabilità nell’uso comune, meglio usare un termine
%         generico che possa poi essere specificato usando l’attributo
%         type (es. <div type=”chapter”>)
%         i <div>, numerati o meno, possono essere ‘nidificati’, nessun
%         problema nel riproporre la struttura editoriale visibile
%     \end{block}

% \end{frame}


\begin{frame}
	\frametitle{Intro Text Encoding Initiative}
	\framesubtitle{Schemi di codifica TEI – Moduli base}
	\addtocounter{nframe}{1}

    \textbf{Alcuni attributi possono essere usati con qualsiasi elemento (v. la classe att.global)}

    \begin{block}{ Attributi globali}
        \begin{itemize}
            \item \textbf{n} un numero o un nome non univoco, possibilmente breve, per identificare un elemento
            \item \textbf{rend} informazioni relative all’aspetto (\textit{originale}!) del testo
            \item \textbf{rendition} simile a \textit{@rend}, ma fa riferimento a elementi
            \texttt{<rendition>} inseriti nell’\texttt{<encodingDesc>} (dentro \texttt{<tagsDecl>})
        \end{itemize}
    \end{block}
\end{frame}

\begin{frame}
	\frametitle{Intro Text Encoding Initiative}
	\framesubtitle{Schemi di codifica TEI – Moduli base}
	\addtocounter{nframe}{1}

    \begin{block}{ Attributi globali}
        \begin{itemize}
            \item \textbf{xml:lang} la lingua del testo contenuto da un elemento
            \item \textbf{xml:id} un identificatore univoco per l’elemento
        \end{itemize}
       \textit{NOTA: in base ai moduli usati nello schema sono disponibili ulteriori attributi globali}
    \end{block}
\end{frame}

\begin{frame}
	\frametitle{Intro Text Encoding Initiative}
	\framesubtitle{Schemi di codifica TEI – Moduli base}
	\addtocounter{nframe}{1}

	\begin{center}
		\includegraphics[width=.9\textwidth]{imgs/esempio-attr.png}
	\end{center}
	% \begin{block}{Esempio}
    %     \texttt{<text>
    %     <body>
    %     \emph{<div n="ch1" type=``chapter''>}
    %     <pb n="1"/>[...]
    %     <p>[...] risulta chiaro se avete letto \emph{<title
    %     rend="underline" xml:lang=``fra''>}Les fleurs du
    %     mal</title> [...]</p>
    %     <p>[...] un grande esempio di <foreign
    %     xml:lang=``fra''>savoir faire</foreign> [...]</p>
    %     [...]
    %     </div>
    %     [ altri div ... ]
    %     </body>
    %     </text>}
    % \end{block}

\end{frame}


\begin{frame}
	\frametitle{Intro Text Encoding Initiative}
	\framesubtitle{Schemi di codifica TEI: Moduli base}
	\addtocounter{nframe}{1}

	\begin{center}
		\includegraphics[width=.9\textwidth]{imgs/esempio-attr2.png}
	\end{center}
	% \begin{block}{Esempio}
    %    \texttt{
    %        <text>
    %     <body>
    %     <div n="ch1" type=``chapter''> <pb n="1"/> [...]
    %     <p n=``1''>[...] descritto altrove (si veda ad
    %     esempio \emph{<ref target=``\#Rossi94''>Rossi 1994</ref>})
    %     [...] </p> [...]
    %     </div>
    %     [ altri div ... ]
    %     <div n="bib" type=``bibliography''>
    %     [...]
    %     \emph{<bibl xml:id=``Rossi94''>
    %     <author>Rossi, M.</author>[...]</bibl>}
    %     [...]
    %     </div>
    %     </body>
    %     </text>}
    % \end{block}
\end{frame}

\begin{frame}
	\frametitle{Intro Text Encoding Initiative}
	\framesubtitle{Schemi di codifica TEI – Moduli base}
	\addtocounter{nframe}{1}

    \begin{block}{Errori frequenti}
        \texttt{<div>} non può essere usato allo stesso livello gerarchico
        di \texttt{<p>}, in altre parole non si può alternare \texttt{<div>} con
        \texttt{<p>}
    \end{block}
    
    \begin{block}{Errore!}
        \texttt{<div> [...] </div>
        <p> [...] </p>
        <div> [...] </div>}
    \end{block}
\end{frame}


\begin{frame}
	\frametitle{Intro Text Encoding Initiative}
	\framesubtitle{Schemi di codifica TEI – Moduli base}
	\addtocounter{nframe}{1}

    \begin{block}{Errori frequenti}
        \texttt{<div>} e tutti gli altri elementi strutturali \textit{puri} non
        possono contenere testo.
       
    \end{block}
    
    \begin{block}{Errore!}
        \texttt{<div>Pippo</div>
        <person>Pippo</person>}
    \end{block}
    

\end{frame}



\begin{frame}
	\frametitle{Intro Text Encoding Initiative}
	\framesubtitle{Schemi di codifica TEI – Moduli base}
	\addtocounter{nframe}{1}

	\begin{block}{Enfasi e termini particolari}
       
        \begin{itemize}
            \item \texttt{<emph>} parole o frasi enfatizzate nel testo. (\texttt{Questo è il <emph>mio</emph> computer!})
            \item \texttt{<foreign>} parola o frase in una lingua diversa. (\texttt{In quel punto entrò il bidello a dare il <foreign xml:lang=``lat''>finis</foreign>}).
        \end{itemize}
    \end{block}

\end{frame}

\begin{frame}
	\frametitle{Intro Text Encoding Initiative}
	\framesubtitle{Schemi di codifica TEI – Moduli base}
	\addtocounter{nframe}{1}

	\begin{block}{Enfasi e termini particolari}
       
        \begin{itemize}
            \item \texttt{<distinct>} “diverso” dal testo perché arcaico, gergale, ecc. (\texttt{Saltò in groppa al <distinct>fido destriero</distinct>})
            \item \texttt{<hi>} elemento generico. (\texttt{<hi rend=``double''>N</hi>el mezzo del cammin di nostra vita.
            Il suo nome è <hi rend=``italic''>Mario Rossi</hi>})
        \end{itemize}
    \end{block}

\end{frame}

\begin{frame}
	\frametitle{Intro Text Encoding Initiative}
	\framesubtitle{Schemi di codifica TEI: Moduli base}
	\addtocounter{nframe}{1}

    \begin{block}{Enfasi e termini particolari}
        \begin{itemize}
            \item  \texttt{<mentioned>} parola o frase menzionata ma non usata.
            (\texttt{Il termine corretto è <mentioned>epigrafe</mentioned>})
            \item \texttt{<soCalled>} parola o espressione da cui ci si distanzia
            (\texttt{il cosiddetto <soCalled>darwinismo sociale</soCalled>})
        \end{itemize}
    \end{block}
    
\end{frame}

\begin{frame}
	\frametitle{Intro Text Encoding Initiative}
	\framesubtitle{Schemi di codifica TEI: Moduli base}
	\addtocounter{nframe}{1}

    \begin{block}{Enfasi e termini particolari}
        \begin{itemize}
            \item \texttt{<term>} una o più parole considerate termine tecnico.
            (\texttt{Possiamo definire il <term xml\:id="NPL" rend=``italic''>neopositivismo logico</term>})
            \item \texttt{<gloss>} una spiegazione o glossa riguardo il testo.
            (\texttt{<gloss target=``\#NPL''>una corrente filosofica basata
            sul principio che la filosofia debba aspirare al rigore
            proprio della scienza </gloss>})
        \end{itemize}
    \end{block}
    
\end{frame}

\begin{frame}
	\frametitle{Intro Text Encoding Initiative}
	\framesubtitle{Schemi di codifica TEI – Moduli base}
	\addtocounter{nframe}{1}

	\begin{block}{Esercizio}
       \textbf{ Marcare un testo plain text di circa 3000 caratteri a piacere.}
        \begin{itemize}
            \item inserire prologo XML
            \item marcare la struttura usando gli elementi fin qui descritti
            in particolare marcare tutti i paragrafi usando \texttt{<p>} e la struttura editoriale usando \texttt{<div>}
            \item verificare che sia ben formato con xmllint
            \item salvare il file XML su github
        \end{itemize}
    \end{block}
\end{frame}

%% sezione relativa alla infrestruttura TEI

% TEI XML focuses on the meaning of text, rather than its appearance.

% TEI XML can be used for a simple reading-oriented transcription of a primary source, whether that be an authorial manuscript, a printed literary work, an audio broadcast, or a dictionary. It can be used for enriched encodings in which many aspects of such texts are made explicit, so that software of all kinds can operate upon them, from visualisation tools and digital publishing systems to specialised statistical analysis packages. It can be used to provide additional annotations and metadata of all kinds.


% deciding on the proper content for that new element does require some knowledge of the way the TEI system is designed

\begin{frame}
    \frametitle{Infrastruttura TEI}
    \framesubtitle{Tabella Moduli TEI}
    \addtocounter{nframe}{1}
    
    \begin{block}{TEI framework}
        
            La tecnologia TEI ha un framework concettuale diviso in
                \begin{itemize}
                    \item Moduli
                    \item Classi
                    \item Macro
                    \item Tipi di Dato
                \end{itemize}

    \end{block}
\end{frame}


\begin{frame}
    \frametitle{Infrastruttura TEI}
    \framesubtitle{Tabella Moduli TEI}
    \addtocounter{nframe}{1}
    
    \begin{block}{Moduli TEI}
        Un modulo è semplicemente un contenitore per una serie di dichiarazioni uniformi e coerenti per gli elementi TEI e le relative classi.
    \end{block}
\end{frame}

\begin{frame}
    \frametitle{Infrastruttura TEI}
    \framesubtitle{Tabella Moduli TEI}
    \addtocounter{nframe}{1}
    
        \begin{center}
        \includegraphics[width=.95\textwidth]{imgs/ModuliTEI.png}
        \end{center}
   
\end{frame}

\begin{frame}
    \frametitle{Infrastruttura TEI}
    \framesubtitle{Tabella Moduli TEI}
    \addtocounter{nframe}{1}
        \begin{center}
        \includegraphics[width=.95\textwidth]{imgs/ModelloTEI.png}
        \end{center}
\end{frame}

\begin{frame}
    \frametitle{Infrastruttura TEI}
    \framesubtitle{Tabella Moduli TEI}
    \addtocounter{nframe}{1}
    
    \begin{block}{Classi TEI}
        Le classi sono usate per esprimere due distinti tipi di \textbf{caratteristiche comuni} ad un insieme di elementi. 
    \end{block}

    \begin{block}{Classi TEI}
        Gli elementi di una classe possono \textit{condividere un insieme di attributi} oppure possono far \textit{parte di uno stesso content model}.
    \end{block}
\end{frame}

\begin{frame}
    \frametitle{Infrastruttura TEI}
    \framesubtitle{Tabella Moduli TEI}
    \addtocounter{nframe}{1}
    
    \begin{block}{Classi TEI}
        \begin{itemize}
            \item Un elemento appartenente ad una classe attributo condivide gli attributi con tutti gli altri elementi membri della stessa classe. 
            \item Un elemento appartenente alla classe modello condivide il luogo del content model dove appare con gli altri elementi membri della stessa classe.
        \end{itemize}
    \end{block}
    \textit{ In entrambi i casi un \textbf{elemento eredita} proprietà dalle classi di cui è membro.}
\end{frame}

\begin{frame}
    \frametitle{Infrastruttura TEI}
    \framesubtitle{Tabella Moduli TEI}
    \addtocounter{nframe}{1}
        \begin{center}
        \includegraphics[width=.95\textwidth]{imgs/Classi-AttributiGlobaliModuli.png}
        \end{center}
\end{frame}


\begin{frame}
    \frametitle{Infrastruttura TEI}
    \framesubtitle{Tabella Moduli TEI}
    \addtocounter{nframe}{1}
    
    \begin{block}{Macro TEI}
        Le Macro sono shortcut per dichiarazioni che occorrono frequentemente. 
        \\ Le Macro sono utilizzate in due modi diversi:
        \begin{itemize}
            \item per content model o parti di content model \textit{frequently-encountered}
            \item per datatype di attributi
        \end{itemize}
         
    \end{block}
\end{frame}

\begin{frame}
    \frametitle{Infrastruttura TEI}
    \framesubtitle{Tabella Moduli TEI}
    \addtocounter{nframe}{1}
    
    \begin{block}{Data Type TEI}
        I valori che possono assumere gli attributi sono definiti da tipi di dato all'interno delle \textit{TEI datatype specification}.
    \end{block}

    \begin{block}{Data Type TEI}
       
    \end{block}
    Le specifiche TEI definiscono i propri tipi di dato sfruttando altri tipi di dato primitivi e quelli derivati dalle specifiche W3C. 
\end{frame}





% xx sezione 1 frame 01
% \begin{frame}
%     \frametitle{Attributi Globali}
%     \framesubtitle{Elenco}
%     \addtocounter{nframe}{1}


% \textbf{\textrm{att.global} provides attributes common to all elements in the TEI encoding scheme.}

% \begin{description}
%     \item [@xml:id]     \textbf{identifier} provides a unique identifier for the element bearing the attribute.
%     \item [@n]          \textbf{number} gives a number (or other label) for an element, which is not necessarily unique within the document.
%     \item [@xml:lang]   \textbf{language} indicates the language of the element content using a ‘tag’ generated according to BCP 47\footnote{see \href{http://google.con}{http://google.com}}.
% \end{description}

% \end{frame}


% % xx sezione 1 frame 02
% \begin{frame}
%     \frametitle{Attributi Globali}
%     \framesubtitle{Elenco cont..}
%     \addtocounter{nframe}{1}


% \textbf{\textmd{att.global} provides attributes common to all elements in the TEI encoding scheme.}

% \begin{description}
%     \item [rend] [att.global.rendition]	(rendition) indicates how the element in question was rendered or presented in the source text.
%     \item [style] [att.global.rendition]	contains an expression in some formal style definition language which defines the rendering or presentation used for this element in the source text
%     \item [rendition] [att.global.rendition]	points to a description of the rendering or presentation used for this element in the source text.
% \end{description}

% \end{frame}

% % xx sezione 1 frame 03
% \begin{frame}
%     \frametitle{Attributi Globali}
%     \framesubtitle{Elenco cont...}
%     \addtocounter{nframe}{1}


% \textbf{\textmd{att.global} provides attributes common to all elements in the TEI encoding scheme.}

% \begin{description}
%     \item [xml:base]	provides a base URI reference with which applications can resolve relative URI references into absolute URI references.
%     \item [xml:space]	signals an intention about how white space should be managed by applications.
%     \item [source] [att.global.source]	specifies the source from which some aspect of this element is drawn.
% \end{description}

% \end{frame}

% % xx sezione 1 frame 04
% \begin{frame}
%     \frametitle{Attributi Globali}
%     \framesubtitle{Elenco cont....}
%     \addtocounter{nframe}{1}


% \textbf{\textmd{att.global} provides attributes common to all elements in the TEI encoding scheme.}

% \begin{description}
%     \item [cert] [att.global.responsibility]	(certainty) signifies the degree of certainty associated with the intervention or interpretation.
%     \item [resp] [att.global.responsibility]	(responsible party) indicates the agency responsible for the intervention or interpretation, for example an editor or transcriber.
% \end{description}

% \end{frame}


% xx sezione 1 frame 05

\begin{frame} [fragile]
    \frametitle{Attributi Globali}
    \framesubtitle{Esempio \textrm{@xml:lang}}
    \addtocounter{nframe}{1}

    \textbf{\textrm{xml:lang} indica la lingua e il sistema di scrittura usato}
    \defverbatim{\langatt}{%
        \begin{tiny}
        \begin{verbatim}
            <TEI xmlns="http://www.tei-c.org/ns/1.0">
                <teiHeader xml:lang="en">
                    <!-- ... -->
                </teiHeader>
                <text xml:lang="fr">
                    <body>
                        <div>
                            <!-- chapter one is in French -->
                        </div>
                        <div xml:lang="de">
                            <!-- chapter two is in German -->
                        </div>
                        <div>
                            <!-- chapter three is French -->
                        </div>
                        <!-- ... -->
                    </body>
                </text>
            </TEI>
        \end{verbatim}
        \end{tiny}
        }
        \begin{center}
            {\langatt}
        \end{center}
\end{frame}


% % xx sezione 1 frame 06
% \begin{frame}
%     \frametitle{Attributi Globali}
%     \framesubtitle{Stile e Aspetto}
%     \addtocounter{nframe}{1}

    
%     \textbf{In the TEI scheme, it is possible to supply information about the appearance of elements within a source document in the following distinct ways:}

%     \begin{itemize}
%         \item One or more properties may be specified as the default for a set of elements (based on an external scheme, by default CSS), using rendition elements and their selector attributes;
%         \item One or more properties may be specified for individual element occurrences, using the rend attribute with any convenient set of one or more sequence-indeterminate tokens;
%     \end{itemize}
% \end{frame}

% % xx sezione 1 frame 07
% \begin{frame}
%     \frametitle{Attributi Globali}
%     \framesubtitle{Stile e Aspetto cont..}
%     \addtocounter{nframe}{1}
    
%     \textbf{Note that these TEI attributes always describe the rendition or appearance of the source document, not intended output renditions, although often the two may be closely related.}

%     \begin{itemize}
%         \item One or more properties may be specified for individual element occurrences, using the rendition attribute to point to rendition elements;
%         \item One or more properties may be supplied explicitly for individual element occurrences, using the style attribute.
%     \end{itemize}

% \end{frame}


% xx sezione 1 frame 08

% \begin{frame}
%     \frametitle{Attributi Globali}
%     \framesubtitle{da vari altri Moduli Tabella}
%     \addtocounter{nframe}{1}
%     %fare una tabella 
% class name	module name	see further
% att.global.linking	linking	16 Linking, Segmentation, and Alignment
% att.global.analytic	analysis	17 Simple Analytic Mechanisms
% att.global.facs	transcr	11.1 Digital Facsimiles
% att.global.change	transcr	11.7 Identifying Changes and Revisions

% \end{frame}

%%%%%%%%%%
% attributi globali: source, cert, resp, xml:base, xml:space
% Altri attributi Globali divisi per classe e per moduli.
% Appendice A per lista alfabetica delle Classi di Modello.

%% Macro e DataType
% Tabella Macro
% 


\begin{frame}
    \frametitle{Infrastruttura TEI}
    \framesubtitle{Classificazione degli elementi}
    \addtocounter{nframe}{1}
    \textit{Quasi tutti gli elementi TEI possono essere \textbf{classificati informalmente} come appartenenti alle seguenti categorie:}
    \begin{block}{TEI element classification}
        \begin{itemize}
            \item divisions
            \item chunks
            \item phrase-level elements
            \item inter-level elements
            \item components
        \end{itemize}

    \end{block}

\end{frame}

\begin{frame}
    \frametitle{Infrastruttura TEI}
    \framesubtitle{Classificazione degli elementi}
    \addtocounter{nframe}{1}
   
    \begin{itemize}
       \item \textbf{divisions}
       \item[] Divisioni ad alto livello dei testi, molto spesso elementi annidati.
    \end{itemize}

   \begin{itemize}
    \item \textbf{chunks}
    \item[] Elementi come i paragrafi e altri elementi simili i quali sono posizionati all'interno dei testi e divisioni. Solitamente non sono elementi che possono annidarsi o apparire all'interno di altri elementi di livello chunk.
    \end{itemize}

\end{frame}


\begin{frame}
    \frametitle{Infrastruttura TEI}
    \framesubtitle{Classificazione degli elementi}
    \addtocounter{nframe}{1}
   

    \begin{itemize}
        \item \textbf{phrase-level elements}
        \item[] Elementi che occorrono solo all'interno di elementi di livello chunk.
     \end{itemize}
 
    \begin{itemize}
     \item \textbf{inter-level elements}
     \item[] Elementi che possono occorrere sia tra chunks all'interno di division, sia all'interno di essi.
     \end{itemize}

     \begin{itemize}
        \item \textbf{components}
        \item[] Elementi che possono occorrere direttamente all'interno dei testi o delle divisioni di testo. E' una combinazione di elementi di livello inter e chunk.
        \end{itemize}
   
\end{frame}


\begin{frame}
	\frametitle{Intro Text Encoding Initiative}
	\framesubtitle{Schemi di codifica TEI – Moduli base}
	\addtocounter{nframe}{1}

	\begin{block}{Struttura di un documento TEI}
        \begin{itemize}
            \item \textit{struttura fondamentale all’interno della radice (\texttt{<TEI>})}
            \item una intestazione TEI (\texttt{<teiHeader>})
            \item un testo: \texttt{<text>} (o più testi, cfr. infra)
        \end{itemize}
    \end{block}
    
\end{frame}


 


\section{Panoramica eXtensible Markup Language}
%\begin{frame}
	\frametitle{Markup language e XML}
	\framesubtitle{soluzione corrente per la codifica dei testi}
	\addtocounter{nframe}{1}

	\begin{block}{XML per la descrizione e la codifica}
		Ad oggi la soluzione considerata ottimale per una corretta rappresentazione del testo è l'adozione dei markup language descrittivi basati su XML.
	\end{block}


\end{frame}

\section{TEI: Codifica Apparato Critico}
\begin{frame}
    \frametitle{TEI Modulo 12 - Codifica Edizioni Critiche}
    \framesubtitle{Getting started}
    \addtocounter{nframe}{1}
    
    \begin{block}{Main Discipline}
    \end{block}

    \begin{block}{Main Goal}
    \end{block}

\end{frame}

\begin{frame}
    \frametitle{Introduction}
    \addtocounter{nframe}{1}
    
    \begin{center}
        \includegraphics[width=.95\textwidth]{imgs/1.png}
    \end{center}

\end{frame}

% This panel addresses the TEI critical apparatus as a data model, investigating how it has expanded the capacity of scholarly editions to articulate and analyze phenomena of textual variation and multiplicity. We will discuss how the TEI critical apparatus, as a structure that mediates distinct versions of a work, is expanding horizons for multidimensional and pluralistic document modeling. Our panel surveys recent experiments with the critical apparatus that have led to new kinds of scholarly research and in some cases to revisions to the TEI Guidelines. What kinds of research questions and applications can we support with the TEI critical apparatus, and what practical challenges do we face in working with it in inline and stand-off ways? We begin by investigating how the TEI critical apparatus has transformed the expressive capacity of scholarly editions to prioritize textual multiplicity. We continue by sharing data models that apply TEI critical apparatus as a stand-off “spine” for connecting independently encoded witnesses. We conclude by inviting the audience to discuss with us the scalability of these methods for texts with large numbers of witnesses, and the technological challenges and opportunities of stand-off methods in light of recent changes to the TEI Guidelines. 



\section{Strumenti per edizioni critiche}
%\input{includes/section3.tex}

\section{Considerazioni Finali}
%
% \begin{frame}
% 	\frametitle{Conclusion and Further Work}
% 	\addtocounter{nframe}{1}
% 	\begin{block}{Conclusion}
		
% 	\end{block}
% 	\begin{block}{Conclusion}
		
% 	\end{block}
% \end{frame}


% \begin{frame}
% 	\frametitle{Conclusion and Further Work}
% 	\addtocounter{nframe}{1}
% 	\begin{block}{Conclusion}
% 		\begin{itemize}
% 			\item 
% 			\item 
% 		\end{itemize}
% 	\end{block}
% \end{frame}


% \begin{frame}
% 	\frametitle{Conclusion and Further Work}
% 	\addtocounter{nframe}{1}
% 	\begin{block}{Conclusion:}
% 		\begin{itemize}
% 			\item<1-> 
% 			\item<2-> 
% 			\item<3-> 
% 			\item<4-> 
% 		\end{itemize}
% 	\end{block}
% \end{frame}

%% lavori futuri %%


\begin{frame}
	\frametitle{Conclusion and Further Work}
	\addtocounter{nframe}{1}
	\begin{block}{FINE DEL SEMINARIO!!}
		\begin{center}
			Grazie per la vostra paziente attenzione
		\end{center}
		\begin{center}
			\textbf{Se ci sono domande..}
		\end{center}
	\end{block}
\end{frame}



\section{References}
%
%bibliografia
\begin{frame}
    \frametitle{References}
    \addtocounter{nframe}{1}
    \begin{thebibliography}{10}
    \end{thebibliography}

\end{frame}

\begin{frame}
    \frametitle{References}
    \addtocounter{nframe}{1}
    \begin{thebibliography}{10}
    \end{thebibliography}

\end{frame}


\end{document}